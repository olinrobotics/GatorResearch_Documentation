\documentclass[12pt]{article}
\usepackage{graphicx} %for embedding images
\usepackage{url} %for proper url entries
\usepackage{hyperref}
\usepackage{float}
%\usepackage{caption}
%\usepackage{subcaption}
\usepackage{pdfpages}



\begin{document}
\renewcommand{\thefootnote}{\arabic{footnote}}

%include other pages
\begin{titlepage}

\begin{center}

\vfill

% Title
\LARGE \textbf {John Deere XUV Gator 850D}\\
\LARGE \textbf {Robot Documentation}\\[0.5in]

\small \textbf{Justin Poh, Claire Diehl, Ryan Eggert, Caleb Kissel} \\
\small \textbf{Academic Year 2015 - 2016}

\vspace{.1in}
Under the guidance of\\
{\textbf{Professor David Barrett}}\\[0.2in]

\vspace{.1in}
Supported by:\\
{\textbf{Student Academic Grant (SAG) \\ Franklin W. Olin College of Engineering}}\\[0.2in]

\vfill

% Bottom of the page
\includegraphics[scale=0.17]{Logos.jpg}
\Large{Intelligent Vehicles Laboratory}\\
\normalsize
\textsc{Franklin W. Olin College of Engineering}\\
Needham, Massachusetts \\


\end{center}

\end{titlepage}


\tableofcontents

\newpage

\listoffigures

\newpage

\listoftables

\newpage
\input{Preface}
\newpage
\section{The Mechanical System}

\subsection{Preface to the Section}
Most of the mechanical components on board the vehicle were installed by the SCOPE team that first worked on the vehicle in 2009 - 2010. As such, many parts of this section will begin with pages from the original SCOPE reports and then followed by supplementary information on what might have changed since the SCOPE report was written. 
%
Today, this mechanical system remains largely the same except for the following changes:

\subsection{The Electronics Box}

The electronics box mounted to the back of the vehicle is, of course, where most of the electronics including the computers, FPGA chassis, the power supplies, and most of the power distribution elements on the vehicle are housed.\\ \\
%
\noindent The electronics box is essentially a heavily modified Nothern Tools Co. metal tool box that is mounted on two vibration isolating rails in the bed of the vehicle. This mounting system consists of two pieces of square channel high tensile strength steel with corrosion resistant coating that is bolted to the bed of the vehicle. The box is then attached to the steel with four vibration-damping sandwich mounts allowing for a relatively smooth ride. This simple system is robust and requires relatively little assembly. The vehicle also includes a roof-mounted optics-quality pegboard for sensor mounting. 

\begin{enumerate}
\item The monitor mentioned in the original SCOPE team report is the Nortec SUN-1710-P daylight-visible monitor. The current monitor on the vehicle is a standard Acer monitor
\end{enumerate}

\newpage

\includepdf[scale=0.9, pages=-, clip, trim=0mm 20mm 0mm 35mm, pagecommand={}]{MechSystem.pdf}

\newpage

\subsection{Additional Mechanical Components}
\newpage
\chapter{The Electrical System}

The electrical system consists of two parts: the power distribution system and signals. Each will be discussed in the following chapters. Wiring diagrams are provided in Appendix.

\section{The Power Distribution System}

\subsection{Power Supplies}
Power to the entire system comes from the 3-prong plug running through the PVC pipe fixtures on the driver's left side of the electronics box.

\begin{figure}[h!]
\centering
\includegraphics[scale=.1]{Photos/PowerStrip.jpg}
\caption{The power strip on the side of the electronics box}
\label{fig:powerstrip}
\end{figure} 

 \noindent The 3-prong plug attached to the power strip plugs into building power when the vehicle is in the Large Project Building and supplies power to the power strip mounted on the wall of the electronics box on the driver's left side. Alternatively, during operation, the 3-prong plug is plugged into a Honda EU2000i generator that should be mounted in the back of the vehicle:\\ \\
%
 Power to the rest of the system is then drawn from that power strip. The devices plugged into that power strip are shown in the diagram below:\\ \\

\begin{figure}[h!]
\centering
\includegraphics[scale=.6]{Photos/PowerStrip_Drawing.jpg}
\caption{Drawing of devices plugged into the power strip}
\label{fig:powerstripdrawing}
\end{figure} 

\noindent The entire system has 3 voltages that powers everything: 5 volts, 12 volts and 24 volts:
\begin{enumerate}
\item A Meanwell HRP-600-24 converts 120V AC power from the power strip to 24V DC power
\item A Meanwell HRP-300-12 converts 120V AC power from the power strip to 12V DC power
\item A Meanwell SB-15B-05 converts 24V DC power from the 24V DC power supply to 5V DC power
\end{enumerate}
%
The 24V and 12V power supplies are located on the lower deck of the electronics enclosure and the 5V power supply is located on the upper deck.

\newpage 

\subsection{Overview of Fuses}
There are 4 main fuse blocks used on the vehicle to distribute power via fuses to all electronic devices on board:
\begin{enumerate}
\item 1 Blue Sea Systems fuse block (C24) handles all 24V power distribution
\item 2 Blue Sea Systems fuse blocks (C12 and M12) handle all 12V power distribution
\item 1 linear fuse block handles all 5V power distribution 
\end{enumerate}
%
From the lower deck, three main power busses run to the upper deck: two 24V busses and one 12V bus. One 24V bus runs to the C24 fuse block and the other 24V bus runs to the 5V power supply that converts 24V DC power to 5V DC power. The 12V bus runs to the M12 fuse block. The C12 fuse block is powered from one of the outputs on the M12 fuse block.
%
\subsection{24V Power Distribution (C24 Fuse Block)}

\begin{minipage}{0.6\textwidth}
Five outputs on the C24 fuse block are used:\\
\begin{enumerate}
\item Signal to a 24V sensing probe
\item LIDAR power
\item LIDAR power
\item Navcomm GPS power
\item Safety light power
\end{enumerate}
\end{minipage} \hfill
\begin{minipage}{0.5\textwidth}
\begin{figure}[H]
\centering
\includegraphics[scale=.05, angle=-90]{Photos/C24.jpg}
\caption{\label{fig:C24} C24 Fuse Block}
\end{figure}
\end{minipage}

\bigskip

\noindent  Except for the signal to the 24V sensing probe, all of the other 4 lines run to the appropriate equipment outside the electronics box. The signal to the 24V sensing probe runs to the voltage sense project box (discussed in a later subchapter).

\newpage

\subsection{12V Power Distribution (C12 Fuse Block)}

\begin{minipage}{0.6\textwidth}
Five outputs on the C12 fuse block are used:
\begin{enumerate}
\item Power for the cab fans
\item Signal to a 12V sensing probe
\item INS power
\item Power for ethernet switch
\item E-Stop 12V power
\end{enumerate}
\end{minipage} \hfill
\begin{minipage}{0.5\textwidth}
\begin{figure}[H]
\centering
\includegraphics[scale=.06, angle=90]{Photos/C12.jpg}
\caption{\label{fig:C12} C12 Fuse Block}
\end{figure}
\end{minipage}

\bigskip

\noindent Power for the cab fans and INS power run to the appropriate equipment outside the electronics box. The signal to the 12V sensing probe runs to the voltage sense project box (discussed in a later subchapter) and power to the ethernet switch runs to the netgear ethernet switch on the back corner of the electronics box on the driver's right side of the vehicle. The E-Stop 12V power is utilized by the E-Stop system, which will be discussed in a later chapter.

\subsection{12V Power Distribution (M12 Fuse Block)}

\begin{minipage}{0.6\textwidth}
All six outputs on the M12 fuse block are used:
\begin{enumerate}
\item Right tilt unit motor power
\item Power to the C12 fuse block
\item Power to the linear actuators
\item Left tilt unit motor power
\item Box fan power
\item Box fan power
\end{enumerate}
\end{minipage} \hfill
\begin{minipage}{0.5\textwidth}
\begin{figure}[H]
\centering
\includegraphics[scale=.06, angle=90]{Photos/M12.jpg}
\caption{\label{fig:M12} M12 Fuse Block}
\end{figure}
\end{minipage}

\bigskip

\noindent The power to the C12 fuse block is obtained from the appropriate port on the M12 fuse block. The power to each of the fans mounted to the front of the electronics box that cools the electronics box is also obtained from 2 of the outputs. Power to the right and left tilt units is also drawn from the M12 fuse block outputs and power then runs to the appropriate equipment outside the electronics box. Finally, the power for the linear actuators runs to an E-Stop relay first, then connected to two fuse terminals on the linear fuse block (discussed on the next page)

\subsection{Linear Fuse Block (Various Power Distribution)}

\begin{minipage}{0.6\textwidth}
The four lines on the linear fuse block are:
\begin{enumerate}
\item Fuse for motor power coming from the steering motor amplifier headed to the steering motor
\item Fuse for the 5V DC power output from the 5V DC power supply headed to the 5V DC power terminal strip
\item Two fuses to distribute 12V power from the E-Stop relay to the gas and brake linear actuators
\end{enumerate}
\end{minipage} \hfill
\begin{minipage}{0.5\textwidth}
\begin{figure}[H]
\centering
\includegraphics[scale=.06]{Photos/5vblock.jpg}
\caption{\label{fig:linear} Linear Fuse Block}
\end{figure}
\end{minipage}

\subsection{Terminal Strips (5V DC Power and Ground)}

There are two terminal strips that serve as busses for 5V DC power and ground.  The 5V DC power is used for:

\begin{enumerate}
\item Both commfronts that process serial signals from the LiDARs
\item To steering encoder
\item To cabin 5 volts
\item Signal to a 5V sensing probe
\end{enumerate}

\noindent The ground terminal strip is used for:

\begin{enumerate}
\item Ground connection for the E-Stop magnet coil
\item Ground to the 5V sensing probe
\item Ground to the E-Stop relay
\end{enumerate}

\newpage

\subsection{Emergency Stop System}

The emergency stop system is designed to mechanically cut power to the actuators in the event of an emergency. The last part of the power distribution system is the Emergency stop system. The system is relatively simple and consists of the following components:\\

\begin{enumerate}
\item 2 red mushroom-head emergency stop switches with black plastic mounting box, 1 located in the cab and one located on the outside wall of the electronics box
\item A magnecraft 788XBXM4L magnet coil relay with 70-463-1 relay socket located on the upper deck of the electronics box
\item The E-Stop sensing probe at the voltage sensing project box (discussed in the next chapter)
\end{enumerate}

\noindent The power source for the E-Stop system is a fused 12V terminal from the C12 fuse block. One branch of thsi line is then run through all the E-Stop switches. There is a second branch of this line that then runs to a relay on the upper deck of the electronics box. This relay controls power to the linear actuators and steering motor amplifier. When the E-Stops are not pressed, the line reads zero volts. When the E-Stops are pressed, the line is pulled up to 12V and that triggers the relay to open the normally-closed circuits and close the normally-open circuits, shutting down power to the linear actuators and disabling the steering motor amplifier.

\section{Subsystems and Signal Connections: Steering System}

The steering system is comprised of the following components:

\begin{enumerate}
\item Potomac Electric DC motor and steering encoder, which is an aftermarket power steering assist motor purchased by the 2009-2010 SCOPE team
\item Advanced Motor Controls 30A20AC analog servo drive with filter card
\item National Instruments NI9263 module that sends +/- 10V control signals to the analog servo drive
\item National Instruments NI9411 module that reads the steering encoder signals
\end{enumerate}

The 30A20AC analog servo driver and filter card are powered directly from the power strip on the side of the electronics box. The 30A20AC analog servo driver and filter card then used to drive the Potomac Electric DC motor. The steering encoder is powered via a 5 volt power line to the encoder.\\ \\
%
\noindent The 30A20AC analog servo driver takes PWM inputs and puts out motor voltage to the steering motor. In the LabVIEW FPGA code, the steering controller code communicates with the NI9263 module and uses the NI9263 module to output PWM control signals to the 30A20AC analog servo drive. Based on those PWM signals, the 30A20AC analog servo drive will then pass the appropriate motor voltage through the filter card before passing those signals to the Potomac Electric DC Motor mounted below the steering wheel. The steering encoder signals are read by the NI9411 module.

\section{Subsystems and Signal Connections: Gas and Brake Actuators}

The gas and brake pedal system consists of the following components:

\begin{enumerate}
\item 2 PQ Controls linear actuators, one for the brake and and one for the gas pedal
\item National Instruments NI9263 chassis that sends analog control signals to the appropriate linear actuators
\end{enumerate}

The linear actuators share a common 12V power line that originates from the M12 fuse block. That power line is then split into two at the linear fuse block so that each linear actuator receives 12V power. \\ \\
%
\noindent The linear actuators accept anywhere between 1 - 4 volts and voltage corresponds linearly to position. Given that the linear actuators both have 3 inch strokes, the conversion should be 1 volt per inch of stroke. The voltage that controls the linear actuators is provided by the NI9263 chassis that is controlled by the FPGA code. 

\section{Subsystems and Signal Connections: LIDAR Serial Communication}

Serial communication with the left and right LIDARs on the front of the vehicle involves the following components:

\begin{enumerate}
\item 2 Sick LMLS291 LIDARs
\item 2 CommFront serial converters, one for each LIDAR
\item National Instruments NI9401 chassis that reads the serial lines
\end{enumerate}

Each LIDAR receives power from a terminal on the C24 fuse block. The CommFront serial converters are needed in order to convert the RS485 serial signal from the LIDARs to an RS232 signal that the NI chassis can interprete. After the CommFront serial converters, the signals are both fed into the NI9401 chassis to be read by the FPGA code.

\section {Subsystems and Signal Connections LIDAR Tilt Units}

The tilt unit system that nods the LIDARs consists of the following components:

\begin{enumerate}
\item 2 National Instruments NI9505 motor controller modules
\item 2 Bodine Electric DC Motors rated for HERP
\end{enumerate}

\noindent Unlike the steering system, the tilt unit system is driven entirely from the NI9505 motor controller modules. The modules themselves receive power from two sources: motor power for each module originates from a terminal on the C12 fuse block and connected to the input V+ and V- terminals on each module. The power for the controller electronics is received from the chassis itself. Each NI9505 motor controller module then drives its associated Bodine Electric motor by putting out power on the M+ and M- terminals. Each NI9505 motor controller module also handles the encoders for its associated tilt unit. The NI9505 motor controller module uses 5 pins on the serial connector on the module. 2 of the pins provide power and ground for each encoder and the other three pins handle the A, B and index outputs of the encoder. NEEDS SIGNAL DESCRIPTION?

\section{Cabin Wire Routing}

There are four areas where cables are routed into the cabin of the Gator. The first two pass-throughs are previously existing holes, located behind the driver-side seat and designed for 3/8? conduit. Conduits for communications with the cabin e-stop button and linear actuator are routed through these holes, and follow pre-existing conduits running along the left roll bar of the Curtis cab, with their last zip-tie fastening in the center of the hood, under the dashboard. The power cables for actuators and steering motor are passed through another pre-existing hole, this one located behind the battery compartment, behind the driver?s side seat and directly above the FWD crankshaft located on the passenger side of the vehicle. The final pass-though is located between the two seats, a half inch above the seat belt mounting brackets. It is not pre-existing, and is drilled to be 1.45? wide, to allow for a VGA cable for the monitor to be passed through. Conduits for monitor power and signal, as well as USB and IEEE 1394 connectivity and signal from the Cherry magnetic sensors (used as end-stops for steering and located in the driver side firewall) in the cabin are passed through this hole.

\newpage

\section{Encoder Signal Connections}

The encoder signal line, including the pull-up resistor breakout boards, all obey the following pin diagrams. On the connectors closer to the encoder, circular connectors are used. When looking at the connector from the inside of the tilt unit housing, the following pin arrangement is observed:

\begin{figure}[H]
\centering
\includegraphics[scale=.3]{Photos/Encoder_6PinCircularPinDiagram.jpg}
\caption{\label{fig:6pincircular} Pin Diagram for 6 Pin Circular Connector for Tilt Unit Encoders}
\end{figure}

\noindent The pins correspond to the following on the encoder:

\begin{table}[H]
\centering
\caption{Pin Designations for Circular Encoder Signal Connector}
\label{pintable}
\begin{tabular}{|l|l|}
\hline
\textbf{A} & A+  \\ \hline
\textbf{B} & -   \\ \hline
\textbf{C} & GND \\ \hline
\textbf{D} & +5V \\ \hline
\textbf{E} & Z+  \\ \hline
\textbf{F} & B+  \\ \hline
\end{tabular}
\end{table}

\newpage

\noindent On the other end of the encoder signal cable is a 9-pin serial connector with the following pin arrangement and designation:

\begin{figure}[H]
\centering
\includegraphics[scale=.3]{Photos/Encoder_9PinSerialPinDiagram.jpg}
\caption{\label{fig:6pinserial} Pin Diagram for 6 Pin Serial Connector for Tilt Unit Encoders}
\end{figure}

\noindent The pins correspond to the following on the encoder:

\begin{table}[H]
\centering
\caption{Pin Designations for Serial Encoder Signal Connector}
\label{pintable}
\begin{tabular}{|l|l|}
\hline
\textbf{1} & A+  \\ \hline
\textbf{2} & B+  \\ \hline
\textbf{3} & Z+ \\ \hline
\textbf{4} & - \\ \hline
\textbf{5} & +5V  \\ \hline
\textbf{6} & -  \\ \hline
\textbf{7} & -  \\ \hline
\textbf{8} & -  \\ \hline
\textbf{9} & GND  \\ \hline
\end{tabular}
\end{table}

\newpage

\begin{table}[H]
\centering
\caption{Pin Designations for Serial Encoder Signal Connector}
\label{pintable}
\begin{tabular}{|l|l|}
\hline
\textbf{1} & A  \\ \hline
\textbf{2} & F  \\ \hline
\textbf{3} & E \\ \hline
\textbf{5} & D\\ \hline
\textbf{6} & B \\ \hline
\textbf{9} & C \\ \hline
\end{tabular}
\end{table}

\section{Networking}
\noindent The vehicle carries the following networking equipment on board in the electronics box:\\
\begin{enumerate}
\item Ubiquiti Networks Rocket M900 radio
\item 2 HGV-906U 900MHz omnidirectional 3dBi antennas, one on the front and one on the rear of the vehicle
\item Two coaxial antenna cables, one 25ft cable running to the front antenna and a 15ft antenna running to the rear antenna
\item Ubiquiti Networks PoE power adapter for the Rocket radio
\end{enumerate}

\noindent The two omnidirectional antennas are each connected to one antenna connector on the Rocket radio. Having two antennas, one on the front and one on the rear of the vehicle maximizes the connectivity and reception signal strength between the vehicle and the other end of the ethernet bridge. The rocket radio is connected via a cat6e ethernet cable to a Ubiquiti Networks PoE power adapter that ships with the radio. A second cat6e ethernet cable then runs from the PoE power adapter to the network switch on board the vehicle.\\ \\
%
\noindent The vehicle can be connected to the Intelligent Vehicle Field Network using the above equipment. See the Intelligent Vehicle Field Network User Manual for further details.

\newpage

\section{Overview of the Software Architecture}
The software that runs the Gator includes a stack running on Robot Operating System (ROS) on the Intel Nuc computer in Linux and a second stack running in LabVIEW on the windows 7 rack computer. The LabVIEW code is designed to be minimally intrusive to any intelligent operation. It's sole purpose is to keep the vehicle safe and prevent the vehicle from sustaining damage. The ROS code, on the otherhand, is designed to have most of the intelligence required for performing missions. This segregation of responsibility therefore facilitates different teams with different mission requirements, allowing any team to run their code on the vehicle with minimal additional set up time. As long as the ROS-LabVIEW interface is obeyed, the software architecture will support swapping out the ROS-based code at any time and the vehicle should still run.\\ \\
%
\noindent The software on the vehicle is broken up into three main parts: Forebrain, Midbrain and Hindbrain. Each part has a different function and operating speed. The hindbrain has the fastest operating speed and is used for low-level control of the control surfaces on the vehicle such as the gas and brake pedals. As such, the hindbrain is implemented on a National Instruments NI FPGA. The midbrain has two parts: one handling LabVIEW internal processing and another handling passing of sensor and vehicle information from LabVIEW to ROS. Finally, the forebrain is entirely ROS-based and handles all the high-level processing tasks such as path planning and intelligent obstacle avoidance. \\ \\
%
In the upcoming sections, the various components of the software stack will be discussed in full detail.

\newpage
\chapter{Software System Definitions}

As in any large-scale system with multiple subsystems, the must be common system-level definitions that are obeyed by all subsystems. In the robot's software, too, there are common definitions obeyed by all layers of code in LabVIEW and in ROS. As such, this chapter will detail the common software system definitions.

\section{The Vehicle Co-Ordinate System}

The vehicle co-ordinate system is defined using the standard used on most ground vehicles: with the positive x direction straight ahead. In order to maintain a right-handed co-ordinate system, the three axes of the vehicle co-ordinate system are therefore defined as:

\begin{enumerate}
\item The positive X axis is pointed to the front of the vehicle
\item The positive Y axis is pointed to the left of the vehicle
\item The positive Z axis is pointed upward
\end{enumerate}

\begin{figure}[h!]
\centering
\includegraphics[scale=.7]{Photos/veh_coord.png}
\caption{Vehicle Co-ordinate System Definition}
\label{fig:veh_coord}
\end{figure} 


\newpage

\noindent The respective rotations in the vehicle co-ordinate system are therefore defined as:

\begin{enumerate}
\item Positive rotation about the X axis (Positive Roll) is roll toward the driver's right side of the vehicle
\item Positive rotation about the Y axis (Positive Pitch) is pitch downward toward the ground
\item Positive rotation about the Z axis (Positive Yaw) is yaw toward the driver's left side (counter clockwise yaw) of the vehicle
\end{enumerate}

\newpage

\section{LIDAR Co-Ordinate Definition}
By default, the Sick LMS290 LIDARs are programmed to transmit distance measurements in millimeters (mm) and angles in degrees with 0 degrees on the right and 180 degrees on the left as shown in the image below:

\begin{figure}[h!]
\centering
\includegraphics[scale=.9]{Photos/LIDAR_AngleDef.png}
\caption[Sick LMS290 Angle Definition]{Sick LMS290 Angle Definition \protect \footnotemark}
\label{fig:sick_angledef}
\end{figure} 
\footnotetext{ Information obtained from Quick Manual for LMS Communication Setup}
\newpage
\chapter{The Robot Software System (LabVIEW)}

\section{The FPGA Hindbrain}
As discussed in the software overview chapter, the Hindbrain of the vehicle is implemented on a LabVIEW FPGA in order to ensure that control loops and essential data processing is done at the fastest possible speeds to reduce system latency even when the vehicle travels at higher speeds. \\ \\
%
The front panel and block diagram of the top-level hindbrain VI is shown below:

\begin{figure}[h!]
\centering
\includegraphics[scale=0.55]{Photos/hindbrainblock_annotated.png}
\caption{Hindbrain top-level VI block diagram with subchapters annotated}
\label{fig:hindbrainblock}
\end{figure} 

\newpage

\noindent As can be seen in Figure \ref{fig:hindbrainblock}, the main subchapters of the FPGA hindbrain are:

\begin{enumerate}
\item Sensing Loop
\item Acting Loop
\item Tilt Unit Drive Status Monitors
\item LiDAR Reading
\item Supporting Hardware Control
\item Relay Control
\end{enumerate}

\subsection{Sensing Loop}

The sensing loop of the FPGA hindbrain essentially passes status information and data from other parts of the FPGA hindbrain code up to the front panel so that the real-time code can access these variables. The block diagram for the sensing loop shown in Figure \ref{fig:hindbrainblock} is shown zoomed in below: 

\newpage

\begin{figure}[h!]
\centering
\includegraphics[scale=1.5]{Photos/sensingloop.png}
\caption{Sensing loop in the hindbrain VI}
\label{fig:sensingloop}
\end{figure} 

\noindent As shown in Figure \ref{fig:sensingloop}, the sensing loop is responsible for exposing front panel elements for the following pieces of data to the real-time code:

\begin{enumerate}
\item Left LiDAR Encoder Position: A debugging indicator that shows the encoder position of the left LiDAR in units of encoder ticks
\item Left Tilt Unit State: A cluster containing an indicator for whether the index had been found, whether the tilt unit has been told to be in initialize mode, whether there is a position error in the tilt unit and the encoder position.
\item Right Tilt Unit State: The same cluster as the left tilt unit state cluster used for the right tilt unit
\item Left LiDAR Position: Another debug indicator that shows the position of the left tilt unit in degrees after the encoder ticks have been converted to degrees
\item Negative Steer Endstop: A boolean that represents whether the negative steer angle endstop has been triggered
\item Positive Steer Endstop: A boolean that represents whether the positive steer angle endstop has been triggered
\item Estop Engaged: A boolean that represents whether either of the physical estop buttons have been triggered
\item Odometer (meters): The distance travelled by the vehicle since the code started running
\item Throttle state: A cluster containing indicators for the gas and brake pedal voltage being sent to the linear actuators for the gas and brake respectively
\item DriveState: A cluster indicating the driving state of the vehicle including the steering wheel angle in degrees, the velocity in miles per  hour and the boolean that represents whether the vehicle should apply the brakes
\item SenseHeartBeat: An indicator that simply provides a blinking light that confirms the while loop is running 
\end{enumerate}

\subsection{Acting Loop}

The acting loop of the FPGA hindbrain essentially performs the opposite function of the sensing loop: to provide indicators for the real-time code to pass commands or instructions down to the FPGA code. The block diagram for the acting loop shown in Figure \ref{fig:hindbrainblock} is shown zoomed in below:

\newpage

\begin{figure}[h!]
\centering
\includegraphics[scale=1.5]{Photos/actloop.png}
\caption{Acting loop in the hindbrain VI}
\label{fig:actloop}
\end{figure}

\noindent As shown in Figure \ref{fig:actloop}, the acting loop is responsible for providing front panel elements for the real-time code to send the following commands to the FPGA code:

\begin{enumerate}
\item Left Tilt Unit Command: A cluster that contains indicators for whether the tilt unit should be in initialize mode, the position setpoint of the tilt unit in degrees, the current output when the tilt unit is in initialize mode and a reset position boolean that indicates whether the position of the tilt units should be reset to zero.
\item Right Tilt Unit Command: The same cluster as Left Tilt Unit Command, but for the right tilt unit
\item Left Tilt Unit Settings: A cluster that contains the controller settings for the left tilt unit including the position proportional, integral and derivative (PID) gains, Current PI gains and the current limit
\item Right Tilt Unit Settings: The same settings cluster as the Left Tilt Unit Settings for the right tilt unit
\item Drive Command: A cluster containing the desired wheel angle in degrees, the desired velocity in miles per hour and a boolean to command the vehicle to apply the brakes
\item Steering Controller Settings: A cluster containing the controller settings for the steering control including the PID gains, upper and lower voltage limits, upper and lower encoder limits, PID to motor conversion constant, motor deadband high and low threshold, a reset position boolean for resetting the steer motor position and two other booleans to tell the vehicle whether to allow positive and negative steering voltages or not.
\item Throttle Controller Settings: Cluster containg the throttle controller settings for the gas and brake controllers including the PID gains for the gas pedal controller, voltage offset that indicates the neutral position voltage, position reset and PID reset boolean, maximum and minimum brake voltage, maximum and minimum gas pedal voltage and the speed of the throttle control loop. 
\end{enumerate}

\subsection{Tilt Unit Drive Status Monitors}
The tilt unit drive status monitors essentially provide inputs and outputs that allow the real-time code to command the NI9505 motor control modules and receive information on the status of the modules. The block diagram for the tilt unit drive status monitors shown in Figure \ref{fig:hindbrainblock} is shown zoomed in below

\newpage

\begin{figure}[h!]
\centering
\includegraphics[scale=1.2]{Photos/tudrivestat.png}
\caption{Tilt unit drive status monitors in the hindbrain VI}
\label{fig:tudrivestat}
\end{figure}

\noindent As can be seen in Figure \ref{tudrivestat}, the left and right drive status loops are basically identical to each other except for the module assignments since the left drive status monitor addresses the NI9505 module for the left tilt unit and the right drive status monitor addresses the other NI9505 module. \\ \\
%
Each drive status monitor loop contains a method for enabling the drive, disabling the drive and monitoring the presence of a power supply (vsup Present), whether the drive has a drive fault (Drive Fault), determining whether the drive is enabled or disabled (Drive Status) and whether the drive is overheating (Over Temperature Fault). By default, the NI9505 modules will be disabled when the vehicle is first powered up and so the drive will need to be enabled in software during initialization. 

\subsection{LiDAR Reading}

Although LiDAR reading shows up on the front panel of the Hindbrain top-level VI as one subVI, the process of reading data from the two Sick LMS291 LiDARs really involves several different subVIs to perform the various tasks. In summary, the process of reading data from the LiDARs involves:

\begin{enumerate}
\item Initializing the LiDARs
\item Read the raw data from the LiDARs
\item Parse and transform the LiDAR data into the vehicle's coordinate system
\end{enumerate}

\noindent On the hindbrain top-level VI, the block diagram for reading the LiDARs shown in Figure \ref{fig:hindbrainblock} is shown zoomed in below:

\begin{figure}[h!]
\centering
\includegraphics[scale=2]{Photos/lidarread.png}
\caption{LiDAR Reading SubVI in the hindbrain VI}
\label{fig:lidarread}
\end{figure}

\noindent As can be seen in Figure \ref{fig:lidarread}, the lidar reading VI takes the offsets for the left (Leonardo) and right (Raphael) LiDARs that provide the translations and rotations that are needed to transform the LiDAR data in the last step of the LiDAR reading process.\\ \\
%
The block diagram of the LiDAR reading VI is shown below:

\begin{figure}[h!]
\centering
\includegraphics[scale=0.75]{Photos/readlidarblock.png}
\caption{Block diagram for the Lidar Reading VI}
\label{fig:readlidarblock}
\end{figure}

\noindent As can be seen in Figure \ref{fig:readlidarblock}, the LiDAR reading VI consists of a flat sequence with three parts:

\begin{enumerate}
\item The first part is an FPGA I/O method node that sets the the input and output lines
\item The second part contains a VI for the left and right LiDAR that sends the commands specified in the LiDAR communication protocol to tell the LiDAR to send data continuously at 25Hz
\item The third part contains two subVIs: a serial read/write VI that writes the required commands to the LiDAR and receives the LiDAR data back and a transformer VI that transforms the LiDAR data coordinate system from the LiDAR's local coordinate system to the vehicle's coordinate system.
\end{enumerate}

\subsection{LiDAR Serial Read/Write}

The block diagram for the LiDAR serial read/write VI is shown below:

\begin{figure}[h!]
\centering
\includegraphics[scale=0.75]{Photos/readlidar_block.png}
\caption{Block diagram for the Lidar Reading VI}
\label{fig:readlidar_block}
\end{figure}

\noindent As can be seen in Figure \ref{fig:readlidar_block}, the VI is essentially a wrapper for the actual FPGA read/write functions that allows the code to reuse the same read/write functions while being able to assign the inputs and outputs to the appropriate LiDARs (either left or right). The VI takes either the left (Leonardo) or right (Raphael) LiDAR name as input and assigns the following variables to the write function:

\begin{enumerate}
\item LIDARLeonardoWriteFIFO: A First In First Out (FIFO) queue that queues the LiDAR commands to be sent to the LiDAR 
\item Chassis1/Mod1/DIO3: The DIO chanel from which LiDAR data is transmitted
\item 543478: The baud rate for LiDAR communication
\item 8: Data bits for the LiDAR
\item None: Parity for the LiDAR
\item  1.0: Stop bits
\item True boolean: Reverse Polarity command for the LiDAR
\end{enumerate}

\noindent The read function is then assigned the following variables:

\begin{enumerate}
\item LeoPos: A FIFO queue that queues the lidar position data to be read by the transformer VIs
\item Chassis1/Mod1/DIO5: The DIO channel from which LiDAR data is received
\item LIDARLeonardoReadFIFO: THe FIFO queue that queues the received LiDAR data
\item 543478: The baud rate for LiDAR communication
\item 8: Data bits for the LiDAR
\item None: Parity for the LiDAR
\item  1.0: Stop bits
\item True boolean: Reverse Polarity command for the LiDAR
\end{enumerate}

\noindent These variables are then passed down to the actual serial write and read functions as shown below:

\newpage

\begin{figure}[h!]
\centering
\includegraphics[scale=0.75]{Photos/readwriteblocks.png}
\caption{Block diagram for the VI providing the actual read/write functions}
\label{fig:readwriteblocks}
\end{figure}

\noindent The serial write VI is a stock serial write VI provided by LabVIEW for the LiDARs that has been modified to include a while loop around the entire serial write sequence for the LiDARs. Therefore, for brevity in this report, the write VI will not be explained in detail. However, for convenience, the documentation for the serial write VI is shown in the figure below:

\begin{figure}[h!]
\centering
\includegraphics[scale=0.75]{Photos/writedocs.png}
\caption{Documentation for the serial write function}
\label{fig:writedocs}
\end{figure}

\newpage

\noindent The block diagram for the serial read VI, then, is shown in the diagram below:

\begin{figure}[h!]
\centering
\includegraphics[scale=0.35]{Photos/serialreadblock.png}
\caption{Block diagram for the serial read function}
\label{fig:serialreadblock}
\end{figure}

\noindent As can be seen in Figure \ref{fig:serialreadblock}, after setting the baud rate, the VI executes a series of steps each time throug the while loop to receive the bits that make up the LiDAR data that is received:

\begin{enumerate}
\item Wait for the start bit of the data
\item Read data
\item Read stop bit
\end{enumerate}

\noindent For brevity, this report is not going to discuss the code that waits for the start bit or the code that reads the stop bit. When zoomed in, the read data chapter of the code is as follows:

\begin{figure}[h!]
\centering
\includegraphics[scale=0.85]{Photos/readdatasection.png}
\caption{Block diagram for the chapter that reads the LiDAR data bits}
\label{fig:readdatachapter}
\end{figure}

\newpage

\noindent As can be seen in Figure \ref{fig:readdatachapter}, the data reading code essentially reads each bit as a boolean and adds it to the array. Once the for loop has been through all the bits in the mesage, it converts it to a number and adds the number to the read FIFO queue that queues up the LiDAR reading to be read by the transforming code. At the same time, code in parallel reads either the left or right encoder position (depending on whether Raphael or Leonardo is selected) and and appends the encoder tick count to a separate position FIFO queue to be read by the transformer code.

\subsection{LiDAR Data Transform}
Referring back to Figure \ref{fig:readlidarblock}, after the LiDAR read/write functions, the next step in processing the LiDAR data is to transform it from the local LiDAR coordinate system to the vehicle coordinate system. 

\subsubsection{Overview of the Transform Process}

The data returned by the Sick LMS290 LIDARs is in the co-ordinate frame of the LIDARs by virtue of the way in which the LIDARs take readings. In order to do useful work with the LIDAR data, the LIDAR data has to be transformed to make it iwth reference to the vehicle co-ordinate system. \\ \\
%
\noindent The transform process has three parts:

\begin{enumerate}
\item Convert each scan from polar co-ordinates to cartesian co-ordinates
\item Rotate the coordinate frames such that the frame local to the LIDAR is in the same orientation as the vehicle co-ordinates
\item Translate the coordinate frames such that the frame local to the LIDAR is translated to line up with the vehicle co-ordinate system
\end{enumerate}

\subsubsection{Constant LIDAR Transform Properties}
Based on the design and placement of the LIDAR mounts, the following properties of the LIDAR transform to t
\newpage
\chapter{The Robot Software System (ROS)}

The ROS system is designed to contain as few customizations as possible. This allows our team and future teams to take advantage of as many pre-existing ROS packages as possible. Note that this section of the report assumes familiarity with ROS, basic ROS functionality, catkin and how to create and use ROS packages.

\section{Preparing to Run the ROS Stack and Setting up the Repository}

The NUC onboard the Gator already has all the packages required to run the software stack in ROS Indigo. However, if you want to run the software stack on another computer, you should try to use ROS Indigo unless the IV Lab has upgraded its fleet. You'll then need the following 3rd-party ROS packages to be installed using apt-get and the package manager (e.g. sudo apt-get install ros-indigo-robot-localization):

\begin{enumerate}
\item Robot Localization: \url{http://wiki.ros.org/robot \textunderscore localization}
\item move base: \url{http://wiki.ros.org/move \textunderscore base}
\item URDF: \url{http://wiki.ros.org/urdf}
\item Robot State Publisher: \url{http://wiki.ros.org/robot\textunderscore state\textunderscore publisher}
\item DWA Local Planner: \url{http://wiki.ros.org/dwahttp://wiki.ros.org/dwa_local_planner localhttp://wiki.ros.org/dwa_local_planner planner}
\end{enumerate}

\noindent Once these packages are installed, you can then download the Gator repository from here: \url{https://github.com/olinrobotics/GatorResearch}. Once downloaded, change directory into the master \textunderscore ws folder and then run the command "catkin \textunderscore make". Once done, run the command "source devel/setup.bash". If desired, you should add this sourcing step to your bashrc file so that you don't have to do this manually every time you open a new terminal.

\newpage

\section{ROS Navigation Overview}

The following diagram shows how the ROS Navigation Stack is designed to be used:

\begin{figure}[h!]
\centering
\includegraphics[scale=.35]{Photos/rosnavoverview.png}
\caption{Overview of the Setup of the ROS Navigation Stack}
\label{fig:rosnavoverview}
\end{figure} 

\noindent As can be seen in Figure \ref{fig:rosnavoverview}, the ROS Navigation Stack is made up of a core (move\textunderscore base) and several supporting packages (transforms, odometry sources, a map server, sensor sources and a base controller). 

\subsection{move\textunderscore base (controller)}

The core of the ROS Navigation Stack is the move\textunderscore base package (the big box in the center of Figure \ref{fig:rosnavoverview} that is installed using the package manager. The move\textunderscore base package is responsible for all operations involving commanding the base to move. As shown in Figure \ref{fig:rosnavoverview}, there are 5 key elements to the move\textunderscore base package:

\begin{enumerate}
\item global\textunderscore planner: The global planner is responsible for planning a path to get from where the vehicle is to its destination, regardless of the distance between the origin and the destination. There are several standard options for global planners that can be used as a global planner and any global planner can be used as long as it adheres to the nav\textunderscore core::BaseGlobalPlanner C++ interface defined in the nav\textunderscore core package.
\item global\textunderscore costmap: The global costmap is responsible for maintaining an understanding of all known fixed obstacles in the world of the vehicle using a concept known as cost. For every move the vehicle makes, there is an associated cost to get there. If there are no obstructions to the vehicle, the cost is relatively low. However, if there are obstacles such as tree stumps or rocks at a location, then the cost to be in that location increases. When the vehicle is planning a global path to get from where it is to where it is going to be, the global planner will try to find a route that minimizes the cost to the vehicle of getting from origin to destination. Note that obstacles and features in the global costmap are not updated in real-time. The global costmap is simply created and set beforehand and referenced during normal operation. 
\item local\textunderscore planner: The local planner, like the global planner, is responsible for path planning. However, the local planner is responsible for path planning at a smaller distance scale to plan new paths around obstacles and other obstructions in real-time. So, the local planner is used to plan paths in the immediate vicinity of the vehicle's current position unlike the global planner that can be used to plan paths over much longer distances. 
\item local\textunderscore costmap: The local costmap, like the global costmap, is responsible for maintaining an understanding of all known obstacles in the world of the vehicle using the concept of cost. However, the key difference between the local and global costmaps is that the local costmap is updated in real-time using LiDAR data from the vehicle unlike the global costmap that remains static during operation. Using the LiDAR data from the vehicle, detected obstacles are added to the local costmap and the vehicle will then use the local costmap to plan a path around the obstacle.
\item recover\textunderscore behaviors: Recovery behaviors are used to help the vehicle recover from path planning errors that might be due to erroneous data or a field of view that is too narrow to see a possible path. However, the standard recover behaviors in ROS were written for a holonomic robot that uses differential drive and can therefore turn in place. Since the John Deere Gator XUV is Ackermann-steered and therefore cannot turn in place, we simply turn these recovery behaviors off.
\end{enumerate}

\subsection{Transforms (Input)}

Transforms are a key component of any robot and they provide information about where things are relative to other things. Ultimately, a chain of transforms should provide all the information necessary to determine the position and orientation of something on the robot relative to the vehicle's coordinate system. In ROS, these transforms are an input to the navigation stack and typically indicate the position and orientation of sensors and other key components in relation to other sensors or key components. For example, a transform would indicate the location and orientation of the LiDARs relative to the origin of the vehicle coordinate system. These transforms are important because sensor data is only useful when it can used in conjuction with other systems. However, to work with other systems all these sensors and systems need to be operating with respect to some reference point and, more importantly, they need to be operating with respect to THE SAME refrence point. So, transforms can help transform sensor data from being with reference to the sensor's own local coordinate system to being with reference to the vehicle coordinate system, which is the coordinate system used by all sensors and systems on the John Deere Gator XUV.\\ \\
%
There are 2 important types of transforms:

\begin{enumerate}
\item Sensor Transforms: these are transforms that indicate the position and orientation of the sensor relative to some reference point. The sensor transforms on the John Deere Gator XUV will be discussed in more detail later.
\item Robot Joint Transforms: these are transforms that indicate the position and orientation of all joints between the different parts of the robot and also indicate what type of joints they are (e.g. fixed, revolute). 
\end{enumerate} 

\noindent Regardless of what type of transform it is, all transforms are written in a file known as a Univeral robot Definition File (URDF). Based on the information in that file, the transforms are published by a package known as "Robot State Publisher", which reads the URDF file and then publishes the numerican value of the transforms. More details on this can be found later in this document.

\subsection{Odometry Sources (Input)}
\label{sec:gatorloc}

Odometry sources are another key input to the ROS Navigation Stack. Unlike transforms, odometry sources provide information about the location of the origin of the vehicle's coordinate system (which moves with the vehicle when the vehicle moves) relative to some fixed (inertial) reference frame that does not move even when the vehicle is moving. Common sources of odometry include wheel odometry for dead reckoning as well as GPS for getting absolute, geospatial information about where the vehicle is.\ \\
%
There are three fixed reference frames used by the John Deere vehicle for navigating:

\begin{enumerate}
\item The Odom Frame: the odom frame is a fixed frame that is typically used as the fixed reference frame for the location of the vehicle during deadreckoning. As such, information provided with reference to the odom frame provides information about the location of the vehicle relative to where it was when the navigation stack was first run. Sensors that contribute to the location of the robot with reference to the odom frame usually provide continuous, uninterrupted streams of data (e.g. IMU and wheel velocity from wheel encoders)
\item The Map Frame: the map frame is a fixed frame that is typically used as the fixed reference frame for location of the vehicle when the vehicle is travelling within a mapped or known environment. If the vehicle starts at the origin of the map, then the odom and map frames will be on top of each other. Otherwise, the origin of the map frame will be wherever the origin of the map was defined during mapping. Sensors that contribute to the location of the robot with referene to the map frame are usually more prone to dropouts (e.g. GPS)
\item The UTM Frame: the origin of the UTM frame is the origin of all GPS coordinates. Thus, information provided with reference to the UTM frame indicates the GPS position of the vehicle
\end{enumerate}

\noindent The vehicle can be localized using any number of sensor sources and, in some cases, having multiple sensors contributing to vehicle localization can help reduce the error in the vehicle's location information. However, using multiple sensor sources requires the ability to fuse those sensor sources to ultimately generate one source of information about the location of the vehicle. So, the John Deere vehicle uses the robot\textunderscore localization package that provides pre-written nodes for fusing sensor sources to obtain odometry information. The robot\textunderscore localization package provides three nodes for obtaining odometry information:

\begin{enumerate}
\item ekf\textunderscore localization\textunderscore node: The EKF localization node uses an Extended Kalman Filter (EKF) to fuse sensor data. On the John Deere vehicle, two copies of this node are run: 
\begin{enumerate}
\item One copy is used to fuse all continuous sources of sensor information: IMU and wheel odometry. The output is the transform from the vehicle's coordinate system (origin on the ground under the rear axle of the vehicle) to the odom frame (wherever the vehicle was when the navigation stack was first run). 
\item Another copy is used to fuse the above odom frame information with GPS information that is susceptible to drop-outs. The output is a transform from the vehicle's coordinate system (origin on the ground under the rear axle of the vehicle) to the map frame.
\end{enumerate}
\item ukf\textunderscore localization\textunderscore node: this node is similar to the ekf\textunderscore localization\textunderscore node except that it uses a slightly different Kalman filter. This node is not used at all on the John Deere Vehicle.
\item navsat\textunderscore transform\textunderscore node: this node uses the raw GPS information as well as IMU information to create a transform between the vehicle's coordinate system to the UTM frame. 
\end{enumerate}

\subsection{Map Server}

The map server is exactly what it sounds like: it is a server that stores a copy of the map that the vehicle should be driving in. Typically, when the vehicle is first powered up, a human operator would need to tell the robot where in the map it started. The robot will then take care of keepign itself localized within that map while driving. Typically, the map is generated using the gmapping package or any other ROS mapping package. On the John Deere vehicle, this has not been implemented at the time of writing.

\subsection{Sensor Sources}

Sensor sources are very important to the vehicle for localization (as discussed above) as well as position confirmation and obstacle avoidance. Data such as IMU, GPS and wheel velocity information are used for localization while LiDAR data is used for obstacle avoidance and to confirm that the majority of the obstacles and features detected by the LiDAR are also features that can be found in the map.

\subsection{Base Controller}

The base controller is also exactly what it sounds like: it is a node that knows how to control the base given the desired heading and velocity commands (also known as cmd\textunderscore vel commands) that the Navigation Stack will produce. On the John Deere vehicle, the entire LabVIEW code serves as the vehicle controller. 

\section{Custom Catkin Packages}

In addition to the standard packages discussed above, the John Deere vehicle also has several custom packages that the team wrote to allow the vehicle to make use of the ROS navigation stack.  These include:

\begin{enumerate}
\item gator\textunderscore odom: Contains a script and launch file that generates the wheel odometry information
\item gator\textunderscore nav: Contains the launch and yaml files that are required to set the settings for the navigation stacks and launch the navigation stack
\item gator\textunderscore msgs: Defines any custom messages that are created specifically for the Gator
\item gator\textunderscore localization: Contains the launch files needed to launch the necessary robot\textunderscore localization and navsat\textunderscore transform nodes to allow the Gator to localize accurately
\item gator\textunderscore description: Contains the launch files, STL files and URDF file that create a visual model of the Gator
\item gator\textunderscore communication: Contains python scripts that listen to all sensor data coming from LabVIEW over UDP and publishes it in ROS to the appropriately named topic
\item gator: An unused package that was set up for the purpose of containing all the highest-level launch files for running the vehicle
\end{enumerate}

Each of these packages will be discussed in more detail in the sections below.

\subsection{gator\textunderscore odom}

The gator\textunderscore odom package contains the following files:

\begin{enumerate}
\item wheel\textunderscore odom.py: Creates a wheel odometry node that reads the odometry calculations made by LabVIEW and uses that odometry information to compute the odometry transform from the origin of the vehicle (i.e. on the ground under the rear axle of the vehicle) to the odom frame.
\item wheel\textunderscore odom.launch: The launch file that launches wheel\textunderscore odom.py as a node in ROS to calculate and provide the odom frame transform. It also specifies the vehicle\textunderscore length parameter that wheel\textunderscore odom.py needs to calculate the odom frame transform.
\end{enumerate}

\subsection{gator\textunderscore nav}

The gator\textunderscore nav package provides all the files needed to setup and run the navigation stack. Since the navigation stack uses standard ROS packages, the gator\textunderscore nav package provides the following parameter files to set the right settings for the standard ROS navigation stack:

\begin{enumerate}
\item costmap \textunderscore common\textunderscore params.yaml: A text file that sets the required parameters that are common to both the local and global costmaps. Full details can be found here: \url{http://wiki.ros.org/navigation/Tutorials/RobotSetup}  in the Costmap Configuration section.
\item global\textunderscore costmap\textunderscore params.yaml: A text file that sets the required parameters specific to the global costmap. Full details can be found at the same URL as the URL provided in bullet point 1.
\item local\textunderscore costmap\textunderscore params.yaml: A text file that sets the required parameters specific to the local costmap. Full details can be found at the same URL as the URL provided in bullet point 1.
\item dwa\textunderscore local\textunderscore planner\textunderscore params.yaml: A text file that sets the parameters for the DWA local planner used on the John Deere vehicle. Full details on the parameters required can be found at: \url{http://wiki.ros.org/dwa\textunderscore local\textunderscore planner#Parameters}.
\item global\textunderscore planner\textunderscore params.yaml: A text file that sets the parameters for the global\textunderscore planner package used as the global planner on the John Deere vehicle. Full details on the required parameters can be found at: \url{http://wiki.ros.org/global\textunderscore planner?distro=jade#Parameters}
\end{enumerate}

In addition, the gator\textunderscore nav package provides the following launch files:

\begin{enumerate}
\item move\textunderscore base.launch: A launch file that launches the standard ROS navigation stack with the parameters provided above. Full details can be found at: \url{http://wiki.ros.org/navigation/Tutorials/RobotSetup#Creating\textunderscore a\textunderscore Launch\textunderscore File\textunderscore for\textunderscore the\textunderscore Navigation\textunderscore Stack}
\item gator\textunderscore config.launch: A launch file that launches the vehicles configuration files including launching all sensors, launching all localization nodes, the state publisher and the wheel odometry publisher. Full details can be found at: \url{http://wiki.ros.org/navigation/Tutorials/RobotSetup#Creating\textunderscore a\textunderscore Robot\textunderscore Configuration\textunderscore Launch\textunderscore File}
\item navigation.launch: A launch file that launches the first 2 launch files together to properly launch the entire navigation stack and required components.
\end{enumerate}

\subsection{gator\textunderscore msgs}

This package simply provides definitions for custom messages used on the vehicle. The only two provided at the time of writing are a Drive State message and a GlobalPose message.

\subsection{gator\textunderscore localization}

This package provides all the localization nodes required for the vehicle. Since the vehicle's localization is carried out by the robot\textunderscore localization package, this package merely provides launch files that does the actual launching. An explaination for the high-level organization of the localization package was discussed earlier in Section \ref{sec:gatorloc}.\\ \\
%
Full details on the parameters speified in the launch file can be found at: \url{http://wiki.ros.org/robot_localization}\\ \\
%
The launch files included are:

\begin{enumerate}
\item ekf\textunderscore loc\textunderscore cont.launch: A launch file that launches a copy of the EKF localization node for handling the continuous data coming from the IMU and wheel odometry. The output is a transform from the vehicle coordinate system to the odom frame.
\item ekf\textunderscore loc\textunderscore gps.launch: A launch file that launches a second copy of the EKF localization node for fusing the output of the continuous data localization node with the GPS data. The output is a transform from the odom to the map frame.
\item navsat\textunderscore transform.launch: A launch file that launches the navsat transform node that filters the raw GPS data
\item localization.launch: A launch file that launches all 3 launch files above to properly launch a complete localization package.
\end{enumerate}

\subsection{gator\textunderscore description}

The gator\textunderscore description package provides all the required information for describing the vehicle, its sensors and other hardware as well as where these items are located on the vehicle. The package provides the following files:

\begin{enumerate}
\item BaseGator.urdf: A URDF file that describes how to assemble a model of the vehicle using the different parts
\item STL Files: STL files for all the different parts of the vehicle that are needed to assemble a full model of it. The STL files include:
\begin{enumerate}
\item The base vehicle
\item Front camera mouting assembly
\item Front peg board
\item Left and right tilt unit
\item left and right LiDAR
\end{enumerate}
\item display.launch: A launch file that launches the robot\textunderscore description package to parse the URDF. The launch file also launches the joint state publisher and robot state publisher that publish the joint and robot states. Finally, the launch file also launches rviz to visualize the vehicle model.
\item gator\textunderscore state\textunderscore pub.launch: A launch file that is exactly the same as the display.launch file, but without the additional rviz launching step. This file is used by the gator\textunderscore nav package to launch the localization elements of the code without also launching rviz.
\end{enumerate}

\subsection{gator\textunderscore communication}

The gator\textunderscore communication package is a large package that provides all the required elements for the ROS code to communicate with the LabVIEW code over UDP. In the package, there are 5 files:

\begin{enumerate}
\item The "launch" file contains:
\begin{enumerate}
\item sensors.launch: A launch file that launches all the nodes to read all the data being transmitted by LabVIEW
\item teleop\textunderscore key.launch: A launch file that launches the nodes required to teleoperate the vehicle via keyboard
\end{enumerate}
\item The "msg" file contains message definitions that are no longer in use but could be used in the future if required.
\item The "scripts" file contains all the scripts that are needed to listen over UDP for the data coming from LabVIEW and re-publish them in ROS on the appropriate topic.
\item The "src" and "srv" files contain example ROS publisher and subscriber files and ROS service definition files respectively. They are not used directly on the vehicle but can be used as templates to create publishers or subscribers or to create a ROS service if needed.
\end{enumerate}

\noindent It is worth noting that this document will not discus full details of each of the scripts that transfers data from LabVIEW to ROS. Full documentation can be found within the scripts themselves.

\subsection{gator}

The gator package was created with the intention of storing high-level launch files that launch the entire vehicle's packages at once. At the time of writing, however, the team did not manage to get these packages written and so this package remains blank for a future team to complete it.
\newpage
\input{SoftwareSystemOCU}

\end{document}
