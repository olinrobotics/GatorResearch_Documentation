\renewcommand\thesection{\arabic{section}}
\section{The Olin Robotic Tuna Research Project}

\subsection{Introduction to the Olin Robotic Tuna Research Project}
The long-term goal of the Olin Robotic Tuna research project is to design a robotic tuna capable of completing a trans-atlantic voyage. To achieve this goal, the project has been divided up into three phases, starting with a one-man launchable robotic tuna and ending with a much larger version capable of housing the required infrastructure for a long trans-atlantic voyage. \\ \\
%
The project is currently still in phase 1. Last summer (summer 2014), we developed a proof of concept version of the robotic tuna to prove that our actuation concept was feasible. In the fall of 2014, we then proceeded to develop some of the software required for autonomous operation of the robotic tuna. This semester, we refocused our research efforts on designing a more efficient verison of the propulsion system for the robotic tuna.
%
\subsection{Why a Tuna?}

The main reason for designing a robotic tuna is propulsion efficiency. Biological tuna fish have a highly efficient method of propulsion that wastes very little energy compared to any existing method of propulsion used on sea vessels today. This efficiency is achieved through several key modifications:
%
\subsubsection{1. Eliminating Water Rotation}
%
Despite the wide variety of boat propellers and motors available, most conventional water propulsion methods essentially rely on a spinning propeller pushing water out the back of the sea vessel in order to propel the sea vessel forward. While this is a great propulsion method, the use of propellers means that, in addition to pushing water out behind the sea vessel, the water is also being rotated in the vertical plane by virtue of being in contact with the rotating propeller blades. However, to propel a sea vessel forward, the only motion that matters is the straight motion of the water out the back of the vessel. This means that all the energy that is spent to rotate the water is wasted energy since the rotational motion is unnecessary. \\ \\
%
In a biological tuna, however, there is no such rotation of the water and so the tuna immediately gains this energy saving.
%
\subsubsection{2. Vortex Manipulation}
%
The second, and more intriguing, reason that a tuna can achieve such high propulsion efficiency is that a tuna detects and manipulates vortices occuring around it in order to increase the propulsion forces acting on it.\\ \\
%
As a tuna swims through the water, small swirls of water, known as vortices, are generated that travel down the side of its body as shown in the figure below:

\begin{figure}[h!]
\centering
\includegraphics[scale=0.13]{Fishswirl2.jpg}
\caption{Vortices along the body of a tuna}
\label{fig:simplemag}
\end{figure}

\noindent When these vortices reach the tail, the tuna positions its tail appropriately such that the tail is in the correct position to be pushed on by the vortices as the vortices leaves the tail of the tuna. This allows the tuna to gain additional propulsion from these vortices. The following diagram\footnote{Diagram from Triatafyllou, M. S. \& Triatafyllou, G. S. (1995). An Efficient Swimming Machine, Scientific American, 272.3, 64-71} illustrates this:

\newpage
\begin{figure}[h!]
\centering
\includegraphics[scale=1]{fishswirl.PNG}
\caption{Vortex shedding from the tail of a fish}
\label{fig:simplemag}
\end{figure}


\noindent Thus, because the tuna is capable of making use of the movement of the water around it to achieve additional propulsion, it is able to further reduce the amount of energy that is wasted in water movement that does not contribute to propulsion.