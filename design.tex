\section{Actuator Design}

For our design, we were inspired by the design of hard disk arm actuators. An image of one is shown below\footnote{Image from \url{http://www.kepcil.com/kepcilin/harddisk/hdslider/bigcoil1.jpg}}:

\begin{figure}[h!]
\centering
\includegraphics[scale=.35]{hddarm.jpg}
\caption{Diagram of a Hard Disk Drive Arm}
\label{fig:hddarm}
\end{figure}

\noindent As can be seen in the figure above, the voice coil is a coil of magnet wire through which a current is passed. The voice coil then sits between a pair of magnets, one positioned above and one positioned below the voice coil. These magnets thus apply a magnetic field across the voice coil. When a current is passed through the voice coil, the voice coil experiences a force which causes it to pivot around the actuator axis and move the arm. \\ \\
%
This seemed to be a good concept for what we expected for the actuator for the fish. The oscilating back and forth motion needed to actuate the sections of the tuna tail could be achieved by swinging the arm back and forth. However, we now needed to figure out how to design such an actuator to be as efficient as possible.

\newpage
\subsection{Desired Motion and Proposed Control Scheme}
The design of the actuator began with deciding on the desired motion we want to achieve and how the control system could be designed to achieve this. For this prototype, the goal was to design an actuator that would allow the actuator arm to oscillate back and forth in a rotary motion. Doing so would therefore allow us to attach the oscillating part of the actuator to a section of the fish tail and therefore oscillate the tail. \\ \\
%
In terms of the control scheme, one can imagine that this desired motion is similar to a child on a swing. The child rides on the swing by swinging back and forth in essentially an oscillatory motion. In order to keep the swing moving, the child must deliver impulses of velocity at some point in the motion by swinging their legs out. Therefore, for the actuator to achieve sustained oscillatory motion, it too must deliver impulses of velocity at some point in its oscillatory motion. \\ \\
%
In terms of a control scheme, however, the key question is when during the oscillation does the impulse of velocity need to occur? To answer this question consider the velocity-position plot of the motion as shown in the figure below:

\begin{figure}[h!]
\centering
\includegraphics[scale=.3]{vel-posplot.jpg}
\caption{Conceptualized velocity-position plot of the actuator as it oscillates.}
\label{fig:velposplot}
\end{figure}

\noindent As can be seen in Figure \ref{fig:velposplot}, after starting at some initial position with some initial velocity at t=0, frictional and other losses will cause the oscillation to decrease in amplitude, assuming no external force is applied to restore the oscillation. As such, the circle traced on the velocity-position plot will shrink in radius as shown by the circle labeled t$>$0. \\ \\
%
To apply a restoring force, one could imagine that the two positions that would allow the velocity impulse to have the largest impact on the motion are:

\begin{enumerate}
\item Deliver the velocity impulse at equilibrium position where position is zero and velocity is maximum; or
\item Deliver the velocity impulse at the extreme positions where velocity is zero and posiiton is maximum
\end{enumerate}

\noindent The effects of each possible choice are shown in the graphs below:

\begin{figure}[h!]
\centering
\begin{subfigure}{.5\textwidth}
  \centering
  \includegraphics[scale=.3]{vel-posplot_impulse1.jpg}
  \caption{Velocity impulse delivered at \\ maximum position, zero velocity.}
  \label{fig:velpos_impulse1}
\end{subfigure}%
\begin{subfigure}{.5\textwidth}
  \centering
  \includegraphics[scale=.3]{vel-posplot_impulse2.jpg}
  \caption{Velocity impulse delivered at \\ maximum velocity, equilibrium position.}
  \label{velpos_impulse1}
\end{subfigure}
\caption{Velocity-position plots of the actuator oscillation with the velocity impulse delivered at different points}
\label{fig:impulseplots}
\end{figure}

\noindent As can be seen in the above figure, applying the velocity impulse at the equilibrium position results in the largest change in oscillation. As such, in the mechanical design of this actuator, the magnets should be placed at the equilibrium position of the oscillation in order to be able to deliver a pulse of velocity at equilibrium to sustain the oscillation.

\newpage
\subsection{Overview of the Mechanical Design}
The prototype of the actuator designed during this research period is shown in the figure below:

\begin{figure}[h!]
\centering
\includegraphics[scale=.3]{fullcadpic.png}
\caption{Render of the prototype actuator}
\label{fig:fullcadpic}
\end{figure}

\noindent The actuator has four main mechanical components:
\begin{enumerate}
\item The iron core, which consists of three pieces: the upper frame, the magnetic shield and the lower frame. The purpose of the iron core is to contain the magnetic flux within it and minimize the loss of magnetic flux into the surroundings. 
\item The voice coil spool that will hold the magnet wire voice coil. When current is passed through that magnet wire, a resulting force will act on the voice coil spool.
\item The actuator arm that holds the voice coil spool. When current is applied to the voice coil, the force experienced by the voice coil spool will move the arm and cause it to swing back and forth
\item The test stand in this prototype represents the supporting structure that will be needed to mount the components that make up the actuator.
\end{enumerate}
%
The following sections will now explain the design of each component in detail.

\subsection{The Iron Core}

The iron core is one of the most important components in the actuator and the actuator could not function without it. As discussed in section 3.2, in order to for the actuator to be as efficient as possible, there must be a loop of ferrous material that is able to capture the magnetic flux produced by the magnet in order to contain it and concentrate it across the air gap where it is required. This minimizes the energy losses to the environment. As such, the iron core makes up part of the structure of the actuator and the magnets that will be used to generate a magnetic field are mounted to this iron core. The figure below shows this configuration:

\begin{figure}[h!]
\centering
\includegraphics[scale=.3]{ironcoreexplode.JPG}
\caption{Exploded view of the iron core}
\label{fig:ironcoreexplode}
\end{figure}

\noindent As can be seen in Figure \ref{fig:ironcoreexplode}, the loop of ferrous material that contains the magnetic circuit is made up of the upper frame, shielding plate and lower frame. The magnets that provide the magnetic flux for the system are then mounted to the upper and lower frame and the voice coil fits around the shielding plate. With this configuration, the magnetic flux from the magnet crosses the air gap either between either the upper frame and shielding plate or between the lower frame and shielding plate. Either way, the flux then ends up in the lower shielding plate where it can then travel back around to the upper or lower frame to form a complete magnetic circuit. As a result, this design should concentrate the magnetic flux in the air gap where the voice coil will be located to maximize the force generated. In addition, because the magnets are mounted on the interior of the iron core, the loss of magnetic flux to the surroundings should be minimized.

\subsection{The Voice Coil}

In this prototype, the dimensions of the voice coil were designed such that the lorentz force generated was enough to counteract the springs that are being used to provide restoring force (see section \ref{actuatorarm} for more details on how springs are involved in the design). As such, it was necessary to calculate the dimensions of the voice coil in order to ensure that the actuator would work as expected. \\ \\
%
To start, the table below lists some of the system parameters that have been set and cannot be changed:

\begin{table}[h]
\begin{tabular}{l|l}
\textbf{Parameter Description}                                    & \textbf{Parameter Value}  \\
Maximum Spring Force                           & 0.76lbf $\approx$ 3.4N          \\
Length of Voice Coil Exposed to Magnetic Field & 2.54cm $=$ 0.0254m \\
Magnetic Field Generated by Magnets            & 6000 Gauss $\approx$ 0.6 Tesla       
\end{tabular}
\caption{Table of set parameters for the voice coil}
\label{voicecoilparam}
\end{table}

\noindent Using Equation \ref{forceeq}, we can then express the force equation as follows:

\begin{equation}
\vec{F}=3.4 = (0.6)(0.0254)NI
\end{equation}

\noindent where N is the number of turns on the coil and I is the current flowing through the magnet wire that will be wound around the voice coil holder. \\ \\
%
Thus,

\begin{equation}
NI=\frac{3.4}{(0.6)(0.0254)} \approx 223.1
\end{equation}

\noindent Therefore, we have two factors in the design of the voice coil that we can manipulate: the number of turns on the voice coil and the current flowing through it. However, the magnet wire has a fixed maximum current rating that should not be exceeded so we do not have absolute freedom over that parameter. \\ \\
%
For this prototype, we are using 20AWG wire which has a maximum current rating of 11 amps under chassis wiring conditions (exposed to air to cool). Given that we want a Factor of Safety (FOS) of 2, therefore, the maximum current load through the wire should be:

\begin{equation}
A_{max}=\frac{11}{2} \approx 5.5\mbox{A}
\end{equation}

\noindent Given that maximum current rating, the number of turns needed to generate sufficient force is: 

\begin{equation}
N=\frac{223.1}{5.5} \approx 41 \mbox{ turns}
\end{equation}

\noindent Therefore, to add additional FOS, we should use 50 turns in the voice coil. \\ \\
%
Using this, we can now calculate physical dimensions of the holder around which the magnet wire will be wound. 20 AWG magnet wire has a wire diameter of 0.032in. If 50 turns were laid out in one single row, the length would be $50(0.032) = 1.6\mbox{in}$. This would be a ridiculous width given the dimensions of the rest of the prototype. However, we can stack multiple layers of coils. As such, if we use a two-layer configuration, the width required for the voice coil holder is now only 0.8 inches, which is much more reasonable. To avoid too tight a fit, the voice coil holder has therefore been designed to be 0.9 inches wide. The other dimensions for the voice coil holder are then set by the mechanial design of the rest of the actuator. \\ \\
%
A picture of the voice coil is shown below:

\begin{figure}[h!]
\centering
\includegraphics[scale=.2]{voicecoil.JPG}
\caption{Picture of the voice coil around which the magnet wire will be wound}
\label{fig:voicecoil}
\end{figure}

\newpage

\subsection{The Actuator Arm} \label{actuatorarm}

To transmit the force acting on the voice coil, the voice coil is then mounted on an actuator arm as shown in the figures below:

\begin{figure}[h!]
\centering
\includegraphics[scale=.2]{armature.JPG}
\caption{Picture of the actuator arm with the voice coil mounted to it.}
\label{fig:armature}
\end{figure}

\noindent In this configuration, the force acting on the voice coil will be transmitted along the actuator arm to the other end of the arm. In Figure \ref{fig:armature}, the other end of the arm contains two mount points for springs. In this prototype, the springs serve as the restoring force to allow the actuator arm to oscillate back and forth. However, when the actuator is installed on the actual robotic fish, the other end of the actuator arm will be connected to a tail section of the fish instead of springs in order to move that tail section. In addition, a 1/4in shaft is used as the pivot point for the actuator arm and the arm is pinned to the shaft using an 1/8in spring pin.

\subsection{The Test Stand}
Finally, the last major component of the actuator is the test stand. In the final actuator that will be installed on the robotic tuna, the test stand will be replaced by a proper waterproof housing that will contain the electronics and mechanical structure. However, for this prototype, it is a test stand as shown in the figures below:

\newpage
\begin{figure}[h!]
\centering
\begin{subfigure}{.5\textwidth}
  \centering
  \includegraphics[scale=.15]{teststand1.JPG}
  \caption{View from the back of the test stand}
  \label{fig:teststand1}
\end{subfigure}%
\begin{subfigure}{.5\textwidth}
  \centering
  \includegraphics[scale=.12]{teststand2.JPG}
  \caption{View from the front fo the test stand}
  \label{teststand2}
\end{subfigure}
\caption{Two different views of the test stand with labeled parts}
\label{fig:teststandviews}
\end{figure}

\begin{figure}[h!]
\centering
\includegraphics[scale=.2]{teststand_explode.JPG}
\caption{Exploded view of the test stand}
\label{fig:teststand_explode}
\end{figure}

\noindent As can be seen in the above figures, two flange bearings provide the support for the weight of the actuator arm with the voice coil attached to it. The test stand also provides mounting points for the springs that will provide the restoring force. The iron core then mounts on the front of the test stand. The key features of the test stand, as shown in Figures \ref{fig:teststandviews} and \ref{fig:teststand_explode}, are:

\begin{enumerate}
\item Mounting holes for 6-32 flat-head machine screws to mount the iron core to the test stand
\item Upper and lower flange bearings and bearing carriers to support the actuator arm. The upper bearing carrier is secured using four 4-40 socket head cap screws.
\item Spring mount points to mount springs that will provide the restoring force to allow the actuator arm to oscillate back and forth.
\end{enumerate}



\subsection{Electrical System and Data Collection}
In addition to the mechanical design, in order for this prototype to actually function and be useful, it needs an electrical system and a way to collect data.\\ \\
%
Given the mechanical design discussed above and the desired control scheme, the electrical system must be capable of delivering a current impulse to the voice coil at the point when the actuator arm is at the equilibrium position. To do this, the actuator must be able to:

\begin{enumerate}
\item Detect when the actuator arm is at the equilibrium position
\item Trigger a relatively large current impulse using a relatively low current and voltage signal
\end{enumerate}

\noindent As such, although not fully designed, the electrical system consists of the following components:

\begin{enumerate}
\item A rotary encoder to detect the position of the actuator arm. For this prototype, we are using a US Digital E4T-250-250-S-H-D-2 rotary encoder.
\item A microcontroller such as an arduino or NI myRIO
\item A solid-state relay rated for at least a 12A throughput current
\item Power supply also rated for at least 12A of throughput current
\end{enumerate}

\noindent With these components, the concept is that the rotary encoder will be mounted on top of the upper bearing carrier as shown in Figure \ref{fig:teststandviews}. The rotary encoder can then provide position information. Using this information, velocity can be calculated in order to plot the velocity-position plot. This information can also be used by the microcontroller to detect when the arm is at the equilibrium position. The microcontroller can then emit a "True" voltage (typically a 5V signal) to the solid-state relay in order to trigger the current impulse required to generate a lorentz force acting on the voice coil. This system, however, has not been tested and future work will include testing this electrical system to determine its efficacy.