\newpage
\section{Conclusion and Future Work}

In conclusion, the goal for research on the robotic tuna this semester was to design a more efficient propulsion system for the tail of the robotic tuna in order to achieve greater propulsion efficiency. By designing an actuator that relies on lorentz forces, we believe that the robotic tuna will be extremely efficient because it is able to harness natural phenomena such as resonance and vortex shedding in order to increase its propulsion efficiency substatially. \\ \\
%
Although the actuator has been designed, it has not actually been prototyped or tested. As such, future work on this project will include prototyping and testing the design to determine its efficiency as well as force-generating capacity. The actuator then needs to be scaled down in size in order to be of appropriate size and weight to be used in the actual robotic fish.